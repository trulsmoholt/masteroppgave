\documentclass[../Main/main.tex]{subfiles}

\begin{document}
	\graphicspath{{../Richards equation/figs/}}
	\chapter{Richards' equation}
	In this chapter we discuss numerical solving algorithms for the Richards' equation \eqref{eq:richards}. We start by discretizing in time, then handling the non-linearities and finally using the elliptic discretizations we discussed in chapter 2 and 3. We consider a simplified version of Richards' equation where the gravity and the non-linearity in the permability have been removed.
	
	\begin{equation}\label{eq:richards_simplified}
		\begin{aligned}[c]
			\partial_t \theta (\psi) - \nabla \cdot \pmb{K} \nabla p &= F \\
			\psi &= g \\
			\psi &= u_0
		\end{aligned}
		\ \ \
		\begin{aligned}[c]
			x &\in \Omega  \\
			x &\in \partial \Omega \\
			x &\in \Omega  
		\end{aligned}
		\ \ \
		\begin{aligned}
			t&\in (0,T] \\
			t&\in (0,T] \\
			t&=0
		\end{aligned}
	\end{equation}
	We expect low regularity in time, so there is not much gained by using a higher order discretization in time. The two choices we have left is the forward euler(explicit) and the backward euler(implicit). The obvious choice is backward euler, as it is stable for long timesteps.\todo{show why?} Let $\left \{ t_n \right \}_n$ be a sequence of $N+1$ evenly spaced numbers from $0$ to $T$ and let $\tau = \frac{T}{N}$ be the timestep. Then we state the semidiscrete version of \eqref{eq:richards_simplified} by exchanging the time derivative by a difference quotient $\partial_t \theta (\psi^n) = \frac{\theta(\psi^n)-\theta(\psi^{n-1})}{\tau}$
	\begin{equation}
		 \theta (\psi^n) - \tau \nabla \cdot \pmb{K} \nabla \psi^n = \tau F^n+\theta(\psi^{n-1})
	\end{equation}
	
	\begin{definition}
		content...
	\end{definition}
	\todo{explain and possibly prove convergence of L-scheme}
	\subsection*{The modified finite element method}
	Let $V_h$ be the finite dimensional test space as defined in the \nameref{galerkin_fem} section, definition \ref{def:linear ansatz}, with $\left \{ \phi_i \right \}_i$ beieng the standard nodal basis. 
	To make our finite element method conservative, we define as in (Cao, Y., Helmig, R. and Wohlmuth, B.I. (2011) \cite{https://doi.org/10.1002/fld.1926}), another interpolation operator.
	\begin{definition}[Piecewise global interpolator]
		Let $\hat{I}_h$ be an operator that maps from the test space to functions that are piecewise continuous on each control volume.
		\begin{equation*}
			\hat{I}_h:C(\Omega)\rightarrow \left \{ v_h \in L^2(\Omega):v_h|_{\Omega_i} = K \right \}
		\end{equation*}
	And
	\begin{equation*}
		\hat{I}_h v = \sum_{i\in \mathcal{N}\setminus \mathcal{N}_D}v(x_i)\hat{I}_h\phi_i(x)
	\end{equation*}
	Where
	\begin{equation}
		\hat{I}_h\phi_i(x)=\left\{\begin{matrix}
			1 & \text{if } x\in \Omega_i\\ 
			0 & \text{otherwise}
		\end{matrix}\right.
	\end{equation}
	\end{definition}
	\todo[inline]{Extend this definition with some figures and for the boundary}
	The linearized system arising from the simplyfied Richards equation \eqref{eq:richards_simplified} now becomes:
	\begin{equation}\label{eq:richards_L}
		\begin{aligned}
			\text{find }\psi^{n,j}&\in V_n \text{ such that }\\
			\left \langle \hat{I}_h \theta(\psi^{n,j-1}),\hat{I}_h v_h \right \rangle &+L\left \langle \psi^{n,j}-\psi^{n,j-1},\hat{I}_h v_h \right \rangle  \\+ \tau \left \langle \pmb{K}\nabla \psi^{n,j},\nabla v_h  \right \rangle &= \tau \left \langle F^n,\hat{I}_h v_h \right \rangle + \left \langle \theta(\psi^{n-1}),\hat{I}_h v_h \right \rangle\\
			\text{for all }v_h &\in V_h
		\end{aligned}
	\end{equation}
	\begin{theorem}
		Assume a homogenous domain discretized with paralellograms.
		Then the MPFA-L method with L-scheme linearization for Richards' equation gives a system that is equivalent to the modified Finite element method \eqref{eq:richards_L}
	\end{theorem}
	\begin{proof}
		If we in equation \eqref{eq:richards_L} test with the basis functions $\phi_i$ and express the solution $u_h$ as $u_h = \sum u_j^* \phi_j $ we end up with the system 
		\begin{equation}
			\begin{aligned}
				A_{i,j} &= L \left \langle \phi_i,\hat{I}_h \phi_j \right \rangle + \tau \left \langle \pmb{K} \nabla \phi_i,\nabla \phi_j \right \rangle \\
				B_i &= \left \langle \tau F^n + \theta(\psi^{n-1})+L\psi^{n,j-1}-\theta(\psi^{n,j-1}),\hat{I}_h \phi_i \right \rangle
			\end{aligned}
		\end{equation}
		
		\begin{enumerate}
			\item $L\left \langle \phi_i,\hat{I}_h \phi_j \right \rangle$ is equivalent to $L\int_{\Omega_i}dx \delta_{ij}$
			\item 
			\item
		\end{enumerate}
	\end{proof}
	\begin{lemma}
		The Bi-linear form in the modified finite element method \eqref{eq:richards_L}
		\begin{equation}
			a_h(v,w) = L\left \langle v,\hat{I}_h w \right \rangle + \tau \left \langle \pmb{K}\nabla v,\nabla w \right \rangle
		\end{equation}
		Is coercive
	\end{lemma}
	\begin{proof}
		content..
	\end{proof}
	\begin{lemma}[First Lemma of Strang, page 155 \cite{Knabner}]
		Suppose there exists some $\alpha>0$ such that for all $h>0$ and $v\in V_h$
		\begin{equation*}
			\alpha \left \| v \right \|^2_1 \leq a_h(v,v) 
		\end{equation*}
		and let $a$ be continuous in $V\times V$. Then there exist some constant $C$ independent of $V_h$ such that
		\begin{equation*}
			\left \| u-u_h \right \|_1 \leq C\left \{ \inf_{v \in V_h}\left \{ \left \| u-v \right \|_1 + \sup_{w\in V_h}\frac{|a(v,w)-a_h(v,w)|}{\left \| w \right \|_1}+\sup_{w\in V_h}\frac{|l(w)-l_h(w)|}{\left \| w \right \|_1} \right \} \right \}
		\end{equation*}
	\end{lemma}


\end{document}