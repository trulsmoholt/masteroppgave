\documentclass[../Main/main.tex]{subfiles}

\begin{document}

\section*{Abstract}
\markboth{}{}

\section*{Acknowledgements}

\chapter*{Introduction}
\addcontentsline{toc}{chapter}{Introduction}
Understanding porous media and how fluid flows through it has many useful applications, for example predicting the spread of some contaminant in an aquifer. Other examples include $\text{CO}_2$-storage, geothermal energy extraction and brain modelling. One way of understanding the processes involved in porous media flow is to describe it using partial differential equations, which may be time dependent, non-linear, degenerate and almost always impossible to solve analytically. We therefore, want numerical algorithms that solve these PDEs approximately, and at the same time, respect important properties of the equations/problems we consider, such as mass conservation and entropy.\par
In this thesis, we focus numerical techniques for solving Richards' equation, introduced in \cite{doi:10.1063/1.1745010}, in two spatial dimensions, i.e., we want to find the pressure head $\psi$ such that
\begin{equation}\label{eq:rihcards_I}
	\frac{\partial \theta(\psi)}{\partial t} - \nabla \cdot (\bm{\kappa} (\theta (\psi))(\nabla \psi + \bm{e}_z)) = f,
\end{equation} 
with boundary and initial conditions. The above equation models groundwater flow in partially saturated reservoirs. The non-linear functions $\theta(\cdot)$ and $\kappa(\cdot)$ are determined experimentally and corresponds to saturation and hydraulic conductivity respectively, see \cite{article} for a commonly used example. The vector $\bm{e}_z$ corresponds to the gravitational force which we will neglect in this thesis, and $f$ represents any sources or sinks.\par
Equation \eqref{eq:rihcards_I} is time dependent, parabolic, involves two non-linearities and possibly degenerate. We discretize in time with an implicit method, and get a non-linear elliptic PDE in each time step. This is then linearized with a robust linearization scheme, the L-scheme \cite{list2016study,10.1016/j.cam.2003.04.008}, leading to a sequence of linear elliptic PDEs in each time step. Next, we approximate the solution to these linear elliptic PDEs with a spatial discretization, which will be the main focus of this thesis.
\par 
When solving \eqref{eq:rihcards_I} we are often additionally interested in transport phenomena, which means that some solute $u$ is transported with the fluid flux,  $\bm{q}$
\begin{equation}\label{eq:transport}
	\frac{\partial u}{\partial t} - \nabla \cdot (\bm{q}u) = 0.
\end{equation}
We therefore want our approximate solution of \eqref{eq:rihcards_I} to render an approximate flux field which is \emph{locally mass conservative}, that is important when solving \eqref{eq:transport}. Whether or not the approximated flux field is locally mass conservative is a property of the spatial discretization. It turns out that the linear Lagrange finite element method, which we will cover in section \ref{sec:galerkin_fem}, does not in general have this property. There are, however, a class of finite element methods called \emph{mixed finite element methods}, which conserve mass locally, but have the disadvantage of having more unknowns, and are not discussed further in this thesis. We are therefore interested in finite volume methods, as they are designed to be locally mass conservative.
\par
There are also other properties that we want our spatial discretization techniques to have. Among these are \emph{monotonicity}, which means that the discretization does not allow for unphysical oscillations in the approximated solution. Another desirable property is the ability to handle complex geometries, i.e., grids that are not orthogonal, and that consists of general quadrilaterals (rough grids). Moreover, a computationally efficient method is desirable, and therefore, methods with smaller stencils are preferred. In \cite{10.1007/s00211-006-0060-z}, the authors show that it if you have a locally mass conservative method, with a small stencil (nine-point cell stencil in two spatial dimensions), that can handle rough grids, you cannot guarantee unconditional monotonicity. There are therefore always a trade-off between various desirable properties when choosing a spatial discretization method for \eqref{eq:rihcards_I}. If we, for example, only model groundwater flow in one kind of soil on a easy domain, we could use an orthogonal grid, and our discretization would be efficient and unconditionally monotone.
\par 
The MPFA (Multi-Point Flux Approximation) L-method, introduced in \cite{https://doi.org/10.1002/num.20320}, is a finite volume method, and will be the main focus of this thesis. It is a compromise between all the properties we want, with a cell stencil usually consisting of only seven points, good monotonicity properties and ability to handle rough grids. We introduce it in section \ref{sec:L-method}, and provide numerical experiments in chapter \ref{chap:numerical results}. 
\par 
Convergence rate estimates for finite volume methods does not come "out of the box", as they do with finite element methods, at least not for non-orthogonal grids. In \cite{Stephansen2012ConvergenceOT}, A. F. Stephansen shows convergence for the MPFA-L-Method by formulating it as a mimetic finite difference method. Another approach, that was successfully applied in \cite{https://doi.org/10.1002/fld.1787} for the MPFA-L-method on a triangular mesh, is to show equivalence with a mixed finite element method. In \cite{klausen2006robust}, Klausen and Winther used the same approach for the MPFA-O-method on a quadrilateral grid. In this thesis, we will show equivalence between the MPFA-L-method on a parallelogram mesh, and a modified linear Lagrange finite element method with triangular elements. After equivalence is obtained, we use the finite element framework to prove convergence. Our approach is similar to the one used in \cite{https://doi.org/10.1002/num.20525}. 
\par
After convergence for the MPFA-L-method is achieved, we will see how it can be applied to obtain a convergence rate estimate for Richards' equation. To achieve this, we use the techniques found in \cite{list2016study} for proving convergence of the linearization scheme, and the techniques in \cite{FlorinTimeConvergence,Pop2002} for convergence of the time discretization. In the end, we reach an $L^2$ error estimate, which we confirm by numerical experiments in section \ref{sec:numerics_richards}.
\section*{Outline}
\addcontentsline{toc}{section}{Outline}
In chapter one \ref{chap:porous media}, we give a brief introduction to flow in porous media. Highlighting the physical principles that leads to Richards' equation \eqref{eq:richards}.
\par 
In chapter two \ref{chap:numerical approximation}, we cover some of the numerical approximation techniques one may use to solve Richards' equation, with an emphasis on spatial discretization methods. We start by introducing function spaces, followed by the weak formulation and its well posedness. Then we introduce the finite element method, how it could be implemented, and the ideas behind proving its convergence. Further, we introduce finite volume methods, then we cover some common finite volume methods; two-point flux approximation, MPFA-O-method and MPFA-L-method. We give a short introduction to time discretization, specifically implicit backward Euler. We end the chapter by discussing iterative methods for solving non-linear problems.
\par
In chapter three \ref{chap:convergence spatial mpfa}, we introduce a way of handling boundary conditions for the MPFA-L-method, and show its equivalence with a modified finite element method. Then we apply standard finite element theory to show convergence for an elliptic problem.
\par 
In chapter four \ref{chap:convergence richards}, we discuss the Kirchhoff transform of Richards' equation, removing the non-linearity in the constitutive law. Then we prove convergence of a fully discretized and linearized scheme to solve the Kirchhoff transformed Richards' equation.
\par 
In chapter five \ref{chap:numerical results}, we present some of the code written for this thesis, and do numerical experiments involving elliptic and time-dependent equations on rough grids.

\end{document}