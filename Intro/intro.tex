\documentclass[../Main/main.tex]{subfiles}

\begin{document}

\section*{Abstract}
\markboth{}{}
In this thesis we study the optimization of iterative schemes as both linearization methods, and as splitting methods for solving non-linear and coupled partial differential equations (PDEs).\ We consider two equations that are describing processes in porous media; Richards' equation, a possibly degenerate, non-linear and elliptic/parabolic equation that models flow of water in saturated/unsaturated porous media, and Biot's equations, a coupled system of equations that models flow in deformable porous media. 

For Richards' equation we compare the numerical properties of several linearization schemes, including the Newton-Raphson method, the modified Picard method and the L-scheme.\ Additionally, we prove convergence of the linearly and globally convergent L-scheme and discuss theoretically and practically how to choose its stabilization parameter optimally in the sense that convergence is obtained in the least amount of iterations. 

The second aim of the thesis is to effectively solve the quasi-static, linear Biot model. We consider the fixed-stress splitting scheme, which is a popular method for iteratively solving Biot's equations. It is  well-known that the convergence of the method is strongly dependent on the applied stabilization parameter. We propose a new approach to optimize this parameter, and show theoretically that it does not only depend on the mechanical properties and the coupling coefficient, but also on the fluid's flow properties. The type of analysis presented in this thesis is not restricted to a particular spatial discretization, but we require it to be inf-sup stable. The convergence proof also applies to low-compressible or incompressible fluids, and low-permeable porous media. We perform illustrative numerical examples, including a well-known benchmark problem, Mandel's problem. The results largely agree with the theoretical findings. Furthermore, we show numerically that for conditionally inf-sup stable discretizations, the performance of the fixed-stress splitting scheme behaves in a manner which contradicts the theory provided for inf-sup stable discretizations. 
\newpage
\section*{Acknowledgements}
First and foremost I would like to thank my supervisors Florin Adrian Radu and Jakub Wiktor Both for their guidance. I have learned incredibly much during the last two years and you have doubtlessly been the greatest contributors. I am especially grateful to you Jakub for always being available to answer my questions and finding bugs in my code. Thank you Florin for suggesting the topics of the thesis and for showing such an interest in my academic life. You are both great mentors.

I would also like to thank the SFB1313 for granting me the "Scholarship Program for Master's Students" allowing me to stay with the CMAT group at the University of Hasselt. The CMAT group at UHasselt, particularly Iuliu Sorin Pop and Carina Bringedal, I thank for hosting me.

Finally, I thank my friends and family, particularly Laila, for being in my life and making me happy (also) when I am not doing mathematics. 
\chapter{Introduction}
\addcontentsline{toc}{chapter}{Introduction}
There are many topics in the field of flow in porous media that are  of great societal relevance. Some examples are groundwater simulations, $\text{CO}_2$-storage, life sciences and geothermal energy. Common to all of them is the need to solve partial differential equations. These equations are often non-linear, coupled, time-dependent and possibly degenerate, and therefore require robust numerical methods to solve efficiently. We consider two cases in this thesis; the non-linear, time-dependent and possibly degenerate Richards' equation, and the coupled Biot's equations.

When solving non–linear, time-dependent equations one could apply an explicit temporal discretization to avoid solving a non-linear system at each time step. However, this often requires the time steps to be smaller than what is beneficial. The other way to approach the problem is with an implicit temporal discretization. This requires the application of a non-linear solver. The most popular of these solvers is the Newton-Raphson method, which provides a very fast way to solve the problem, but its local convergence property requires a new bound on the time step size. Moreover, it involves computation of derivatives which might be a costly process. Another alternative is to use a globally convergent fixed-point type solver. One example is the L-scheme, in which one includes a stabilization constant instead of the derivatives in the Newton-Raphson method. While this scheme might converge at a slower speed it has several benefits making it competitive to the Newton-Raphson method. 

In this thesis we discuss the theoretical convergence properties of the L-scheme when applied to a special case of two-phase flow; Richards' equation. This equation models flow of water in saturated/unsaturated porous media. In the unsaturated region one assumes that the air moves freely, and therefore its pressure is zero. Hence, the system can be reduced to the single equation describing solely the complementary phase
 
\begin{equation}
\partial_t (s_w(p))-\nabla \cdot \left(\kappa(s_w(p))(\nabla p - \bm g)\right)=f,
\end{equation}
which was proposed by L.A. Richards in 1931 \cite{richards}. The equation and its coefficients are introduced in Section \ref{sec:richardsintro}. Already at this point it is worthwhile to notice that the equation contains two nonlinear terms, $s_w$ (saturation) and $\kappa$ (permeability), which when applying an implicit temporal discretization, like implicit Euler, requires the use of a linearization scheme. The most widely used schemes in the literature are the Newton-Raphson method \cite{lucaputti, lehmann1998}, the modified Picard method \cite{celia} and the L-scheme \cite{list, pop}. While we focus on the theoretical optimization of the L–scheme, we also compare the performance of the L–scheme with the aforementioned schemes. Moreover, we carry out this comparison with the modified L–scheme \cite{koondi} and a localized version of the L–scheme. We introduce the localized version of the L-scheme, where we compute its stabilization parameter for each element, in Chapter \ref{sec:richards}.

We also consider the most commonly used mathematical model for flow in deformable porous media, the quasi-static, linear Biot model (see e.g.\  \cite{coussy}):

Find $({\bf u}, p) $ such that
\begin{eqnarray}%\label{Biot}
-\nabla \cdot (2 \mu \varepsilon( \bm u) + \lambda \nabla\cdot {\bm u I})+\alpha \nabla p &=& {\bm f}, \label{eq:mechanicsintro} \\ [1ex]
\frac{\partial}{\partial t}\left(\frac{p}{M} + \alpha \nabla \cdot {\bm u}\right) - \nabla \cdot (\kappa(\nabla p - {\bm g}\rho))& = &S_f, \label{eq:flowintro}
\end{eqnarray}
where \eqref{eq:mechanicsintro} models balance of linear momentum and \eqref{eq:flowintro} models mass conservation of the fluid. There are two widely used approaches for solving coupled equations: monolithically or by using an iterative splitting algorithm. The former has the advantage of being unconditionally stable, while the latter is much easier to implement, typically building on already available, separate numerical codes for porous media flow and for mechanics. On the other hand, a naive splitting of Biot's equations will lead to an unstable scheme \cite{Kim}. To overcome this, one adds a stabilization term in either the mechanics equation (the so-called {\it undrained split scheme} \cite{kimundrained}) or in the flow equation (the {\it fixed-stress splitting scheme}  \cite{settari1998}). The splitting methods have very good convergence properties, making them a valuable alternative to monolithic solvers for simulation of the linear Biot model, see e.g.\ \cite{settari1998,Kim,andro,jakubAML}. In Chapter \ref{sec:Biot} we discuss the fixed-stress splitting scheme, but we remark that a similar analysis can be performed for the undrained split scheme.

The initial derivation of the fixed-stress splitting scheme had a physical motivation \cite{settari1998,Kim}: one fixes the (volumetric) stress i.e.\ imposes $$K_{dr}{\nabla \cdot \bf u}^i - \alpha p^i = K_{dr} \nabla \cdot {\bf u}^{i-1} - \alpha p^{i-1}$$ and uses this to replace $\alpha {\nabla \cdot \bf u}^i$ in the flow equation. Here $K_{dr}$ is the physical, drained bulk modulus, defined as $K_{dr}=\frac{2\mu}{d}+\lambda$. The resulting stabilization parameter $L$, from now on called on the {\it physical} parameter, is $L_{phys} = \dfrac{\alpha^2} {K_{dr}}$. The physical parameter depends on the mechanics and the coupling coefficient. Consequently, $L_{phys}$ was the recommended value for the stabilization parameter, and one assumed that the method is not converging (it is not stable) for $L < L_{phys}$. In 2013, a rigorous mathematical analysis of the fixed-stress splitting scheme was for the first time performed in \cite{andro}, where the authors show that the scheme is a contraction for any stabilization parameter $L \ge  \dfrac{L_{phys}} {2}$. This analysis was confirmed in \cite{jakubAML} for heterogeneous media, using a simpler technique. A natural question arises immediately: is now $L_{phys}$ or  $\dfrac{L_{phys}} {2}$ the optimal stabilization parameter, in the sense that the number of iterations is smallest? The question is relevant, because the number of iterations to achieve convergence can differ considerably depending on the choice of the stabilization parameter \cite{bause,jakubAML,jakubuwe,mikelicwang}. 

In a recent study \cite{jakubuwe}, the authors considered different numerical settings and looked at the convergence of the fixed-stress splitting scheme. They determined numerically the optimal stabilization parameter for each considered case. This study, together with the previous results presented in \cite{mikelicwang} and \cite{jakubAML} suggest that the optimal parameter is actually a value in the interval $\left[\dfrac{L_{phys}} {2}, L_{phys} \right]$, depending on the data. In particular, the optimal parameter depends on both the boundary conditions and the flow parameters and is not solely dependent on the mechanics and coupling coefficient.

In this thesis we derive a formula, depending on the mechanical parameters, the coupling coefficient and the flow parameters, for choosing the optimal stabilization parameter for the fixed-stress splitting scheme where the values  $\dfrac{L_{phys}} {2}$ and $L_{phys}$ are obtained as limit situations. We prove first that the fixed-stress splitting scheme converges linearly and then derive a theoretical optimal parameter by minimizing the rate of convergence. 
The proof techniques in \cite{jakubAML} are improved to reach the new results. For this we require the discretization to be inf-sup stable which effectively allows us to control errors in the pressure by those in the stress. A consequence of our theoretical result is that the fixed-stress splitting scheme also converges in the limit case of low-compressible fluids and low-permeable porous media. Finally, we perform numerical computations to test the optimized parameter. In Section~\ref{sec:numerics} we find that the numerical results are confirming the theory. In particular, we remark the connection between inf-sup stability and the performance of the fixed-stress splitting scheme: a not inf-sup stable discretization leads to non-monotonic behavior of the splitting scheme with respect to the problems parameters (e.g.\ the permeability).

\section*{Outline}
\addcontentsline{toc}{section}{Outline}
Chapter \ref{sec:basics} contains introductory expositions to various topics which will be used throughout the thesis. Specifically, Section \ref{sec:iterativeschemes} gives an introduction to the basics of iterative schemes, both in the sense of linearization methods and splitting methods. We give basic definitions of convergence properties and present the Banach fixed-point theorem. We then introduce the schemes that we use in Chapter~\ref{sec:richards} and give some information on their stability and rate of convergence. Finally, we discuss two ways to solve coupled equations. 

Section \ref{sec:FEM} introduces the finite element methods, which are applied to all spatial discretizations in the later analysis.  We give a short introduction to Sobolev spaces and prove the Lax-Milgram theorem for existence and uniqueness for variational problems. Then the Galerkin method is defined and at last we present the conforming finite element method which is later used for the numerical tests.

A brief introduction to the basic equations and language of porous media is provided in Section \ref{sec:porousmedia}. We present the energy/pressure relations, the mass balance equation and Darcy's law of flow in porous media. Most importantly we define the Richards equation and the Biot equations which we consider in Chapter \ref{sec:richards} and \ref{sec:Biot}, respectively.

We begin the analysis in Chapter \ref{sec:richards}. Here, we present a spatial discretization, using conforming finite elements, and a temporal discretization using implicit Euler, of Richards' equation. We then analyze both theoretically and numerically the convergence of the L-scheme applied to Richards' equation. Furthermore, a comparative study of the Newton-Raphson method, the modified Picard method, a locally defined L-scheme, the modified L-scheme and the L-scheme is provided. 

In Chapter \ref{sec:Biot} we analyze the fixed-stress splitting scheme applied to the Biot equations. We first present the discretization, conforming finite elements with P1 elements for the flow equation and P2 elements for the mechanics equation as spatial discretization, and implicit Euler for temporal discretization. The fixed-stress splitting scheme is defined and a convergence proof is provided. We derive a formula for how to optimally choose the stabilization parameter of the splitting scheme. Moreover, we discuss the importance of inf-sup stability of the numerical discretization. Finally, we present a numerical study both testing the theory on the optimality of the stabilization constant and the impact of a stable discretization. 




\end{document}