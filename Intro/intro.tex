\documentclass[../Main/main.tex]{subfiles}

\begin{document}

\section*{Abstract}
\markboth{}{}

\section*{Acknowledgements}

\chapter{Introduction}
Understanding porous media and how fluid flows through it has many useful applications, for example predicting the spread of some contaminant in an aquifer. Other examples include $\text{CO}_2$-storage, geothermal energy and brain modelling. One way of understanding the processes involved in porous media flow is to describe it using partial differential equations, which may be time dependent, non-linear, degenerate and almost always impossible to solve with pen and paper. We therefore want numerical algorithms that solve these PDE's approximately, and at the same time respect important properties of the equations/problems we consider.\par
In this thesis we focus numerical techniques to solve Richards' equation, 
\begin{equation}\label{eq:rihcards_I}
	\frac{\partial \theta(\psi)}{\partial t} - \nabla \cdot (\bm{\kappa} (\theta (\psi))(\nabla \psi + e_z)) = f
\end{equation}
which models groundwater flow. 
It is time dependent, parabolic, with two non-linearities and possibly degenerate. Because it is parabolic we discretize in time with an implicit method, and get a non-linear elliptic PDE for each time step. This is then linearized with a robust linearization scheme, leading to a sequence of linear elliptic PDE's for each time step. Next, we solve these linear elliptic PDE's with a spatial discretization, which will be the main focus of this thesis.

\end{document}